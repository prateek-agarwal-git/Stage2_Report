This portion of the problem statement took the maximum effort and time.
The inherited code was never cleaned and had various issues which are discussed below.
\section{Naming of Variables}
\subsection{Issues}
\begin{itemize}
	\item \textbf{Wrong Case} camelCase should be used in all the 5GCore network functions. PascalCase is OK too. 
	However snake\_case, snake\_camelCase were also rampantly used in the RAN code. 
	\item \textbf{Redundancy} Having test or dpdk in the name of every function is not helpful when it is already
	a testing/experimental setup and DPDK APIs are used. 
\end{itemize}
\subsection{Resolution}
All new functions that were defined do not have these issues. The name of many past functions are also changed.
The complete revamp is avoided as past functions might be familiar to the team by the old names.

\section{Stale Comments}
\subsection{Issues}
\begin{itemize}
	\item \textbf{TODOs} There were many todos lying around from very long. 
	The one who would have these TODOs on their list might have completed them on their branch 
	or have never bothered after writing the TODOs.
	\item \textbf{Misleading Comments} The comments were not updated with the change/removal of the lines of 
	the code.
\end{itemize}

\subsection{Resolution}
Whatever TODOs and misleading comments that came in my way have been erased.
\section{Dead Code}
\subsection{Issue}
\begin{itemize}
	\item There were many functions in the code which were never called in any of the data forwarding mode.
	      Some of them were aptly identified as beta functions and many were not.
	\item Hundreds of lines of code was defined in the  conditional statement $if(0)$- they were never called/compiled.
\end{itemize}
\subsection{Resolution}
Dead code mentioned in the issue sections is cleaned.

\section{Deprecated Options}
\subsection{Issue}
There were 20 modes of data forwarding available. Hardly 3-4 were used for testing purposes. Many options 
were redundant as same options were implemented using dpdk APIs and kernel based functionalities.
\subsection{Resolution}
The number of modes are reduced to 5. They are appropriately categorized as main data forwarding modes, helper 
functions and QoS related modes. The code that is not based on DPDK APIs is removed for clarity.
\section{Reused Code}
\subsection{Issues}
DPDK provides a lot of examples showing the usage of their API. It is quite extensive and easy to read and excerpts of 
code can directly be used in our applications. The code should be cleaned when it is lifted from these applications.
The declaration of functions which are never used should be removed. The variables should be named according to
the convention that is followed in our source. The header dpdk\_ran.h had around 600-800 lines of declared functions
which were never defined and used.
\subsection{Resolution}
Most of the declarations discussed here are removed. There might be some declarations in other header files which were not removed. 
Future developers may clean whenever they come across them.
\section{Global Variables}
\subsection{Issue}
Around 150 variables, data structures were declared globally in ranMain.cpp. The use of
global variables made the code highly coupled and made it difficult to change one procedure
without affecting the other. Infact some of them were declared but never used.
This made it also difficult to discern the scope of variable usage inside a procedure - whether is s global variable,
local variable or a class member.
\subsection{Resolution}
\begin{itemize}
	\item The unused global variables are deleted. The variables which were called in only a few functions are made
	      local and passed as parameter.
	\item The remaining global variables are defined in a new namespace \textbf{Global}. This helps in identifying
	      global variables in the procedures' definitions.
\end{itemize}
\section{Directory Structure}
\subsection{Issues}
\begin{itemize}
	\item The earlier RAN directory was not created as a different folder in 5GCore folder as other NFs. 
It was defined inside AMF.  This was very misleading and suggested coupling between AMF and RAN when 
there never was. It was primarily done to avoid creating a new CMakeLists and reuse the build functionality
of AMF.
\item All log files were generated in the parent folder itself. This makes it difficult to read, delete
, transfer log files. 
\end{itemize}

\subsection{Resolution}
\begin{itemize}
	\item A different folder DPDK\_RAN is created inside 5GCore directory.
	\item Log files are now generated in subdirectories. Two log files are generated in each run - throughput 
	log files and debugging information related log files.
\end{itemize}
\section{Poor Refactoring}

\subsection{Issues}
\begin{itemize}
	\item The main function had a switch statement for different modes of operation. This switch statement was approximately 2000 lines long.
	\item The DRY (Don't Repeat Yourself) principle was violated multiple times. If all modes require setting of time duration of the run, it is better to
	define a procedure asking for time rather than repeating it 20 times.
	\item The files were unusually long. The ranMain.cpp and dpdk\_ran.cpp were more than 10k,  8k lines of code respectively.
	The file in which main routine is defined should be small enough for readers of the code to grasp.
\end{itemize}	
\subsection{Resolution}
The code was extensively refactored. Differnt procedures were defined for each of the switch statement option.
Different modes were refactored to make them shorter and easy to understand. 
ranMain.cpp is bifuracted into ranMain.cpp and ran.cpp (for lack of a better name). The header ranMain.h is defined for
inclusion into both the translation units.
Further refactoring can be done of the data forwarding procedures if required. It will be a good exercise in 
understanding of the code.

\section{Unnecessary Offloads}
\subsection{Issues}
As some sections of the code were directly copied from dpdk applications, there were 
some offloads which are unrelated to our application - VLAN, QINQ and MACSEC offloads.
The entire code in dpdk\_ran.cpp was infested with these offloads. These offloads were present with IP
and UDP checksum related offloads which were used by our application.
\subsection{Resolution}
These lines of code were removed from the dpdk\_ran.cpp . However if the surgeon is not an expert,
one may remove healthy and important part with tumors as well. Thankfully, it was possible here 
to reinstall the healthy part back. This took enormous amounts of time and it was a good learning experience
for future.

\section{Technical Improvements}
\subsection{Issues}
\begin{itemize}
	\item High throughput of data plane latency packets. 100 Mbps of data plane latency packets were sent for measuring the 
	end to end latency besides the load. Law of large numbers kicks in for much smaller values and high latency packet throughput 
	causes unnecessary callback overhead. This can further help in moving different functions running on different cores to a single core. 
	This increases the number of cores available for data forwarding.
	\item The statements like \textbf{if (argc == 0)}, for loop running once, comparison of floating point number with equality ($==$), calling an internal function of the
	dpdk API when external call. 
\end{itemize}

\subsection{Resolution}
The mentioned throughput was reduced to 0.4 Mbps. This can be further reduced if required.
The incorrect statements that I came across were removed.
