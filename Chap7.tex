
%\begin{figure}[htbp]
%    \centering
%    \includegraphics[width=0.7\textwidth, keepaspectratio]{./fig/c7f1.png}
%    \caption{VPP Demonstration \cite{barach2018high}}
%    \label{figure17}
%\end{figure}
%

%\begin{figure}[htbp]
%    \centering
%    \includegraphics[width=0.7\textwidth, keepaspectratio]{./fig/c7f2.png}
%    \caption{Multiloop and prefetching with N = 2.}
%    \label{figure27}
%\end{figure}
%
%
%% Figure 2. Multiloop and prefetching with N = 2. caption
%\begin{figure}[htbp]
%    \centering
%    \includegraphics[height = 0.1\textheight, keepaspectratio]{./fig/c7f3.png}
%    \caption{Branch Prediction}
%    \label{figure37}
%\end{figure}
%\subsection{Features}
%    \begin{itemize}
%        \item TLDK implements the main DPDK concepts such as bulk-packet processing, non-blocking API, no context or mode switch, cache and memory alignment.  
%        \item TLDK is built over DPDK and is compatible with vectorized packet processing (\ref{vpp7}). TLDK can not be used with BSD sockets (Linux Socket API).
%        \item \textbf{Pull vs. Push} Network stacks are generally push-based systems - packets are pushed to the application even when they do not need it. TLDK builds a pull-based system in which applicatiion requests for packets when it needs them.
%    \end{itemize}
%Figure \ref{figure47} shows the movement of packets in the DPDK-VPP-TLDK-Application stack. Note that all components are loaded 
%in the same address space. So multiple-memory copies are not required. VPP layer is optional. Data packets move to the application layer. Control packets (like ICMP, ARP requests etc.) are handled below the application layer.  
%
%    \begin{figure}[htbp]
%        \centering
%        \includegraphics[width=0.7\textwidth, keepaspectratio]{./fig/c7f4.png}
%        \caption{Packet movement in DPDK \cite{tldkimg} }
%        \label{figure47}
%    \end{figure}
% % \begin{figure}[p]
% % \centering
% % \subfigure[]{
% % \fbox{\includegraphics[width=0.6\textwidth]{./fig/2d.png}}
% % \label{fig:st5}
% % }
% % \subfigure[]{
% % \fbox{\includegraphics[width=70mm,height=50mm]{./fig/2ddist.png}}
% % \label{2ddist}
% % }
% % \subfigure[]{
% % \fbox{\includegraphics[width=70mm,height=50mm]{./fig/2dcurvature.png}}
% % \label{fig:st6}
% % }
% % \subfigure[]{
% % \fbox{\includegraphics[width=70mm,height=50mm]{./fig/2dth.png}}
% % \label{fig:st7}
% % }
% % \subfigure[]{
% % \fbox{\includegraphics[width=71mm,height=50mm]{./fig/2dth_prev.png}}
% % \label{fig:st8}
% % }
% % \caption{\subref{fig:st1} 2-D Path. \subref{2ddist} Variation of minimum distance of approach between manipulator and workspace obstacles along the path. \subref{fig:st2} Joint curvature comparison. \subref{fig:st3} Joint motion using method I. \subref{fig:st4} Joint motion using method II.}
% % \label{fig:2d}
% % \end{figure}


% % \begin{table}[!ht]
% % \centering

% % \caption{Path length analysis for 2-D workspace} 
% % \footnotesize{
% % \begin{tabular}{|c|c|c|c|c|c|c|}
% % \hline
% % \multicolumn{2}{|c|}{\textbf{Path Length}}&\multicolumn{2}{|c|}{\textbf{Optimal Path Length}}&\multicolumn{3}{|c|}{\textbf{\%Decrease}}\\
% % \hline
% % \textbf{Method}&\textbf{Method}&\textbf{Method}&\textbf{Method}&\textbf{Choice of End}&\textbf{Monotonic}&\textbf{Total} \\
% % \textbf{I}(rad)&\textbf{II}(rad)&\textbf{I}(rad)&\textbf{II}(rad)&\textbf{Configurations}&\textbf{Optimality}& \\[0.5ex]																							
% % \hline
% % 1.040&4.161&1.040&2.603&37.56\%&37.99\%&75.55\% \\[0.5ex]
% % \hline
% % \end{tabular}
% % }
% % \label{2d}
% % \end{table}
% % In order to compare the quality of the paths obtained from the two methods, path length and smoothness comparisons have been done. In the following discussion, decrease in path length from method II to method I due to the two aforementioned factors has been identified separately. Here, the \emph{optimal path length} refers to the path length of a monotonically optimal path having the same start and end configurations as the actual path. It has been used to calculate the contribution of monotonic optimality in decreasing the path length. To compare the smoothness of the paths we present the variation of curvature in joint space $\kappa$, defined as,
% % \begin{equation}
% % \kappa(t) = \frac{\left|\left|\textbf{q}^{\prime\prime}(t) - \left(\hat{\textbf{u}}.\textbf{q}^{\prime\prime}(t)\right)\hat{\textbf{u}}\right|\right|}{\left|\left|\textbf{q}^{\prime}(t)\right|\right|^{2}},
% % \end{equation}


% % \subsection{Staggered Obstacles}
% % \begin{figure}[p]
% % 	\centering
% % 	\subfigure[]{
% % \fbox{\includegraphics[width=0.6\textwidth]{./fig/34.png}}
% % \label{fig:st1}
% % }
% % \subfigure[]{
% % \fbox{\includegraphics[width=70mm,height=50mm]{./fig/34dist.png}}
% % \label{stdist}
% % }
% % \subfigure[]{
% % \fbox{\includegraphics[width=70mm,height=50mm]{./fig/34curvature.png}}
% % \label{fig:st2}
% % }
% % \subfigure[]{
% % \fbox{\includegraphics[width=70mm,height=50mm]{./fig/34_th.png}}
% % \label{fig:st3}
% % }
% % \subfigure[]{
% % \fbox{\includegraphics[width=70mm,height=50mm]{./fig/34_th_prev.png}}
% % \label{fig:st4}
% % }
% % \caption{\subref{fig:st1} 3-D Path. \subref{stdist} Variation of minimum distance of approach between manipulator and workspace obstacles along the path. \subref{fig:st2} Joint curvature comparison. \subref{fig:st3} Joint motion using method I. \subref{fig:st4} Joint motion using method II.}
% % \label{fig:stgrd}
% % \end{figure}
% % \begin{table}[!ht]
% % \centering
% % \caption{Path length analysis for Staggered Obstacles} 
% % \footnotesize{
% % \begin{tabular}{|c|c|c|c|c|c|c|}
% % \hline
% % \multicolumn{2}{|c|}{\textbf{Path Length}}&\multicolumn{2}{|c|}{\textbf{Optimal Path Length}}&\multicolumn{3}{|c|}{\textbf{\%Decrease}} \\
% % \hline
% % \textbf{Method}&\textbf{Method}&\textbf{Method}&\textbf{Method}&\textbf{Choice of End}&\textbf{Monotonic}&\textbf{Total} \\
% % \textbf{I}(rad)&\textbf{II}(rad)&\textbf{I}(rad)&\textbf{II}(rad)&\textbf{Configurations}&\textbf{Optimality}& \\[0.5ex]
% % \hline
% % 3.208&8.807&3.208&5.466&25.64\%&37.93\%&63.57\% \\[0.5ex]
% % \hline
% % \end{tabular}
% % }
% % \label{stg}
% % \end{table}

% % \begin{figure}[p]
% % 	\centering
% % 	\subfigure[]{
% % \fbox{\includegraphics[width=0.6\textwidth]{./fig/shelfinal.png}}
% % \label{fig:st9}
% % }
% % \subfigure[]{
% % \fbox{\includegraphics[width=70mm,height=50mm]{./fig/shelvedist.png}}
% % \label{sheldist}
% % }
% % \subfigure[]{
% % \fbox{\includegraphics[width=70mm,height=50mm]{./fig/shelcurvature.png}}
% % \label{fig:st10}
% % }
% % \subfigure[]{
% % \fbox{\includegraphics[width=70mm,height=50mm]{./fig/shelfinal_th.png}}
% % \label{fig:st11}
% % }
% % \subfigure[]{
% % \fbox{\includegraphics[width=70mm,height=50mm]{./fig/shelprev_th.png}}
% % \label{fig:st12}
% % }
% % \caption{\subref{fig:st9} 3-D Path. \subref{sheldist} Variation of minimum distance of approach between manipulator and workspace obstacles along the path. \subref{fig:st10} Joint curvature comparison. \subref{fig:st11} Joint motion using method I. \subref{fig:st12} Joint motion using method II.}
% % \label{fig:shel}
% % \end{figure}
% % \begin{table}[!ht]
% % \centering
% % \caption{Path length analysis for Shelves}
% % \footnotesize{ 
% % \begin{tabular}{|c|c|c|c|c|c|c|}
% % \hline
% % \multicolumn{2}{|c|}{\textbf{Path Length}}&\multicolumn{2}{|c|}{\textbf{Optimal Path Length}}&\multicolumn{3}{|c|}{\textbf{\%Decrease}}\\
% % \hline
% % \textbf{Method}&\textbf{Method}&\textbf{Method}&\textbf{Method}&\textbf{Choice of End}&\textbf{Monotonic}&\textbf{Total}\\
% % \textbf{I}(rad)&\textbf{II}(rad)&\textbf{I}(rad)&\textbf{II}(rad)&\textbf{Configurations}&\textbf{Optimality}& \\[0.5ex]																							
% % \hline
% % 1.849&5.201&1.849&2.085&4.53\%&59.91\%&64.45\% \\[0.5ex]
% % \hline
% % \end{tabular}
% % }
% % \label{shel}
% % \end{table}
% % \subsection{Bars}


% % \begin{figure}[htbp]
% % 	\centering
% % 	\subfigure[]{
% % \fbox{\includegraphics[width=70mm,height=55mm]{./fig/bars.png}}
% % \label{bar1}
% % }
% % 	\subfigure[]{
% % \fbox{\includegraphics[width=70mm,height=55mm]{./fig/barsdist.png}}
% % \label{bardist}
% % }
% % 	\subfigure[]{
% % \fbox{\includegraphics[width=70mm,height=50mm]{./fig/barsth.png}}
% % \label{bar2}
% % }
% % 	\subfigure[]{
% % \fbox{\includegraphics[width=70mm,height=50mm]{./fig/barscurvature.png}}
% % \label{bar3}
% % }

% % \caption{\subref{bar1} 3-D Path. \subref{bardist} Variation of minimum distance of approach between manipulator and workspace obstacles along the path. \subref{bar2} Joint motion using method I. \subref{bar3} Joint curvature plot.}
% % \label{fig:br}
% % \end{figure}

