5G forwarding plane and the relevant network functions are 
briefly reviewed in \ref{sec5GfwdPlane}. GTP protocol and its 
significance is discussed in \ref{secGTP}. 
The layout of the report is outlined in \ref{secOrganization}. 

\section {5G Forwarding Plane \label{sec:5GfwdPlane}}
The major difference in  data packet processing between 5G and earlier standards is control-user plane 
separation and the use of network function virtualization. Forwarding of packets (user plane), 
authentication of mobile devices (control plane), session establishment and management (control plane) are some of the network 
functions required in the core of a telecommunication network. These network functions run on different or same physical machines as 
virtual machines (preferably) for easier migration/scaling.

This project is mainly concerned with data forwarding plane. The network functions in our implementation will 
run as separate processes. The network functions relevant for forwarding plane are described further.
\subsection{Radio Access Network (RAN) \label{sec:RAN}} 
RAN is a point of contact for all the user equipments (UEs) like handsets, IOT devices, industrial machine controllers etc. 
RAN runs on all the mobile towers and UEs communicate with the one in their vicinity. RAN is
 responsible for talking to Access Mobility Function for authenticating the UEs, registering the  new
  session. The session establishment request is further forwarded to Session Management Function
  (SMF) which establishes a new session and forward session information to the User Plane Function (UPF). 
  \subsection{User Plane Function (UPF) \label{sec:UPF}}
  User plane function (UPF) is responsible for forwarding packets from user equipments to the
   Internet and vice versa. The uplink direction is defined as the flow of the packets from user
    equipments to the Internet. The downlink direction is defined as the traffic coming from the Internet  to the user equipments/RAN. 
  
 \begin{figure}[htbp]
    \centering
    \includegraphics[width=0.7\textwidth, keepaspectratio]{./fig/Introduction/5GFirst.png}
    \caption{5G Forwarding Plane}
    \label{fig5Gforwarding}
\end{figure}
\subsection{Data Network Name (DNN) \label{sec:DNN}}
This network function is the gateway to the public Internet. All incoming packets from outside the
 local network are received by this NF and are subsequently forwarded to the user equipment through
   the UPF and the RAN. 

\section {PFCP Protocol\label{sec:PFCP}}
PFCP stands for Packet Forwarding Control Protocol. Session Management Function (SMF) 
interacts with the User Plane Function (UPF) to setup sessions related information at the UPF.
This  information enables UPF to identify data packets of different sessions coming from RAN 
and provide forwarding, usage report, charging, buffering, QoS related service to the sessions.
Each session is identified by a unique session ID which helps in differentiating among different
UEs/sessions. 
\section {Organization \label{sec:Organization}}
The chapter \ref{chapterUpfArchitecture} discusses the architecture of User Plane
 function's implementation. Chapter \ref{chapterRSS} discusses the  hardware based
  redirection of the packets to different cores. The issues with the standard RSS
 and alternative solutions are also discussed. Chapter \ref{chapterModelsofExecution} discusses the different models of execution used in the implementation of the UPF. Chapter \ref{chapterExperimentsandResults} discusses the experiments performed and results obtained.

