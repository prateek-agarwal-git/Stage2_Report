%Recent technological advancements have encouraged several diverse application domains to provide mobile services. 
Emerging 5G applications require a dataplane that has a high forwarding throughput and low processing latency, in addition to low cost and power consumption. 
%Certain mobile services require ultra-low end-to-end user plane latencies, ranging from {\em{10ms}} to {\em{50ms}}, which the state-of-the-art UPF solutions do not support. 
To meet these requirements, the state-of-the-art 5G User Plane Functions (UPFs) are built over high performance packet I/O mechanisms like the Data Plane Development Kit (DPDK), and further offload some functionality to programmable dataplane hardware. In this paper, we design and implement several standards-compliant UPF prototypes, beginning with a software-only DPDK-based UPF, progressing to designs which offload different functions to programmable hardware. We evaluate and compare the performance of these designs, to highlight the costs and benefits of these offloads. Our results show that offload techniques employed in prior work help improve performance in certain scenarios, but also have their limitations. Overcoming these limitations and fully realizing the power of programmable hardware requires offloading more complex functionality than is done today. Our work presents a preliminary implementation towards a comprehensive programmable dataplane-accelerated 5G UPF. 

%In this paper, we identify the UPF functionalities that can be accelerated using the programmable network hardware. We discuss various offload choices along with the advantages and limitations of each option. We also identify the challenges in the offload and discuss the possible solutions. Our initial prototype evaluation demonstrated up to XXX\% reduction in end-to-end latency, and we estimate up to XXX\%  of power savings.