$n1$ sessions are established before the data forwarding takes place. These sessions remain established throughout the run.

$n2$ sessions are established, modified and released while the data forwarding is also taking place. The data packets are sent from all the currently established sessions. The minimum value of currently established sessions is $n1$. The maximum value is $n1+n2$.

$t1$ is the total duration of the experiment.
$t2$ is the  duration of the establishment, modification and release cycle of each of the $n2$ sessions.

The duration $t1$ for which all the static sessions $n1$ and the dynamic sessions $n2$ are used is also asked to the user.

Data forwarding starts from the $n1$ sessions at the start. After sleeping for a time (currently 5
seconds), $n2$ threads are started. The role of each thread is to sleep for a random amount of time
$t3$ ($<t2$), establish and modify the session, and then sleep for some time $t4$ and release the
session. The invariant is $t3+t4= t2$.  The session is available for data forwarding in the period $t4$.  The threads with new session Ids are started once all the previous threads have joined i.e. have finished their task.
Each of the data packet forwarding cores use all the existing established sessions/UEs to forward the data. Note that this is different from the case when sessions were partitioned among cores.
