5G technology aims to satisfy the increasing requirements of the customers of mobile service 
in terms of higher throughput and low processing latency. 
The major difference in  the processing of data and signalling packets in the core network between 5G and earlier standards is control-user plane 
separation and the use of network function virtualization. Forwarding of packets , 
authentication of mobile devices, session establishment and management  are some of the network functions required in the core of a telecommunication network. 
These network functions run on different or same physical machines as 
virtual machines for easier migration/scaling. 

This project is mainly concerned with the data forwarding plane. The network functions in our setup ran as separate processes.
The high speed data plane is required to meet the ever increasing demands of higher bandwidth and lower latency. The Linux kernel stack cannot meet this demand as it is a general purpose stack catering to different protocols and there is a packet copy overhead from kernel space to the user space for reading the packet payload. The technqiues used to overcome this limitation of the Linux stack is the use of kernel bypass frameworks like Data Plane Development Kit (DPDK), and the use of programmable NICs to offload processing functions in the hardware. 

DPDK runs in software and there is a single copy of packets from NIC buffer to the
 user space. DPDK libraries are optimized for high speed data processing. The use
  of superpages, lockless data structures and customized device drivers for
   various NICs are among some of the salient features of DPDK framework which
   enables high speed processing of data.